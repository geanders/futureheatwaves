% !TeX root = RJwrapper.tex
\title{Working with daily climate model output data in R and the
\pkg{futureheatwaves} package}
\author{by G. Brooke Anderson, Colin Eason, Elizabeth A. Barnes}

\maketitle

\abstract{%
Research on climate change impacts can require extensive processing,
especially when using ensemble techniques to incorporate output from
different climate models and different simulations of each model. This
processing can be particularly extensive when identifying and
characterizing multi-day extreme events like heat waves and frost day
spells, as these must be processed from model output with daily time
steps. Further, climate model output is in a format and follows
standards that may be unfamiliar to most R users. Here, we provide an
overview of working with daily climate model output data in R. We then
present the \pkg{futureheatwaves} package, which we developed to ease
the process of identifying, characterizing, and exploring multi-day
extreme events in climate model output.
}

\section{Introduction}\label{introduction}

Research on climate change impacts can require extensive processing of
climate model output data. This is not only because output files for a
single climate model can be large, but also because of the rising
popularity of ensemble techniques \citep{IPCCch1} in which, to better
characterize uncertainty in projections, impacts are assessed for
multiple climate models, multiple simulations of each climate model, and
multiple climate experiments. Such ensemble techniques help characterize
uncertainty in projections of regional climate change over the next
century due to three distinct sources: (1) internal climate variability,
i.e.~climate noise, (2) climate model uncertainty, i.e.~the same forcing
can produce a different response in different models and (3) scenario
uncertainty, i.e.~uncertainty in future climate forcings (e.g.
\citet{hawkins2009potential}).

A key source of data for ensemble techniques is the Coupled Model
Intercomparison Project, phase 5 (CMIP5; \citet{taylor2012overview}).
This project brought together dozens of major climate modeling groups
around the world to simulate the same future radiative forcing
scenarios, but with their own models. This created an ensemble of
state-of-the-art climate model projections that allow researchers to
study projections and their uncertainties. Most of these modeling groups
additionally performed more than one simulation for each scenario and
model (i.e.~multiple ensemble members), perturbing the initial
conditions by a very tiny amount to quantify uncertainties due to
internal climate variability.

We begin this article with an overview of CMIP5 climate model output
data for R users, focusing on output with a daily time step. We outline
where data from CMIP5 can be obtained as well as how to work with the
file format (netCDF) from R. We overview some R packages that can be
useful when working with this data, as well as aspects of the data
(e.g., non-standard calendars) of which users should be aware when
working with daily climate model output in R.

After this overview, we present the \pkg{futureheatwaves} package, which
we created to aid in identifying and characterizing any type of
multi-day extreme event from daily climate model output (Table
\ref{tab:goals}). The impacts of multi-day extreme events must be
assessed using output in daily time step, unlike other climate impacts
that can be assessed using climate model output at monthly, seasonal, or
yearly time steps. Further, extreme events are identified based on
conditions that are rare for a certain location (e.g., 98th percentile
of local temperature distribution for identifying heat waves)
\citep{IPCCch1}. In this case, assessment is further complicated, as the
event definition must be determined at each study location from climate
model output. Finally, it is often of interest to create summaries of
multiple characteristics of these extreme events. For example, one may
be interested in determining whether the frequency or characteristics
(e.g., length, intensity) of heat waves or warm spells will change under
certain climate change scenarios \citep{IPCCch1}.

The \pkg{futureheatwaves} package handles these challenges and can be
used to identify and characterize a variety of multi-day extreme events
across different ensemble members of one or more climate models. It also
provides some functionality particularly useful in identifying and
characterizing heat waves specifically. Quantification of the impacts of
heat waves on human health suffer from additional sources of uncertainty
beyond those inherent in projections of regional changes in surface
temperature. These include: (1) uncertainty in the definition of a heat
wave itself and (2) uncertainty in the ability of communities to adapt
to changing temperatures (the adaptation scenario). This package
therefore allows the user to create and use a custom extreme event
definition to identify events in the climate model output, as well as
options to explore different scenarios of adaptation to heat.

\begin{table}[t]
\begin{center}
\newcolumntype{L}[1]{>{\raggedright\arraybackslash}p{#1}}
\begin{tabular}{lL{\dimexpr0.8\textwidth-2\tabcolsep\relax}}
\toprule
 & Design goals of the \pkg{futureheatwaves} package \\
\midrule
1 & Make processing of large ensembles of climate simulations more practical for researchers exploring the potential impacts of heat waves and other multi-day extreme events.\\
2 & Speed up processing time by incorporating C++ in event identification.\\
3 & Keep track of the names of climate models and number of ensemble members processed for each.\\
4 & Not only identify, but also characterize, all extreme events within each climate simulation, to allow the exploration of patterns in these characteristics across different simulations and also to allow the use of more complex impacts models, including models that incorporate event characteristics (e.g., event length, event intensity). For example, this package allows the user to apply a health effects model where risk of mortality is not the same for every heat wave, but rather is modified by heat wave length, intensity, or other measured characteristics.\\
5 & Give users extensive power in customizing the process, including allowing custom extreme event definitions.\\
6 & Allow users to easily explore the extreme events identified within all climate simulations.\\
7 & Create output that is in a ``tidy" data format, allowing it to work well with \pkg{ggplot2} for visualization.\\
\bottomrule
\end{tabular}
\end{center}
\caption{Design goals for the \pkg{futureheatwaves} package.}
\label{tab:goals}
\end{table}

\section{An overview of climate model output for R
users}\label{an-overview-of-climate-model-output-for-r-users}

\subsection{CMIP5 climate model output
data}\label{cmip5-climate-model-output-data}

For climate impact studies, a main source for climate model output is
the Coupled Model Intercomparison Project, which is currently in its
fifth phase (CMIP5). Over 20 climate modeling groups created one or more
climate models which, for this project, were run using standardized
scenarios \citep{taylor2012overview}. The resulting output is uniform
across modeling groups and has a consistent structure, which allows
comparison of simulations from different models \citep{IPCCch9}. CMIP5
climate model output is archived at a number of different time steps
(e.g., daily, monthly, seasonal, yearly) \citep{taylor2010cmip5}, and
some variables are reported at multiple levels in the ocean or
atmosphere (e.g., ocean temperature). Here, we will focus on data with a
daily time step for variables reported at a single level (e.g.,
near-surface air temperature).

Each modeling group ran simulations under several experiments, with
experiments varying in terms of radiative forcing by using different
scenarios of time-varying model inputs (greenhouse gas emissions or
concentrations, land use changes, etc.)
\citep{taylor2012overview, IPCCch9}. Experiments include historical
experiments (run using radiative forcing consistent with observed and
reconstructed data for 1850--2005), pre-Industrial control experiments,
and experiments of future scenarios of radiative forcing over the 21st
century or longer (e.g., RCP4.5, RCP8.5) \citep{taylor2012overview}.
Some modeling groups created ensembles of output for a specific model
and experiment, in which they ran the experiment multiple times with the
model with very small changes to the initial conditions.

The CMIP5 climate model output data are distributed across data nodes at
different climate modeling centers \citep{taylor2012overview}, but can
be accessed centrally at the World Climate Research Programme CMIP5 data
portal at \url{https://pcmdi.llnl.gov/search/cmip5/}. Users must
register before downloading data, and some data are restricted to
non-commercial use. There is a separate file for each combination of
climate model, experiment, modeling realm (e.g., atmosphere, ocean),
variable, time step, and ensemble member
\citep{taylor2012overview, taylor2010cmip5}. For finer time scales, the
output is further split across multiple files for specific year ranges
(e.g., 5 years of output for each file) \citep{taylor2010cmip5}. Each
file's name includes the output variable, climate model, experiment, and
ensemble member for the simulation \citep{taylor2010cmip5}.

CMIP5 files can be searched and downloaded through a point-and-click web
interface. They can also be downloaded in bulk to computers with Unix or
Mac operating systems using the \code{wget} file downloading utility.
Appropriate \code{wget} scripts can be created either through the World
Climate Research Programme CMIP5 data portal or through Earth System
Grid Federation's Search RESTful API. Tips on efficiently searching and
downloading the data, including through use of wget scripts and the
search API, are available as user tutorials through the website of the
University of Colorado Boulder's Earth System CoG (e.g.,
\url{https://www.earthsystemcog.org/projects/cog/doc/wget} for a
tutorial on downloading files using wget).

CMIP5 files are saved in Network Common Data Format (netCDF), a binary
file format that allows storage of data representing a regular array.
For climate model output at a single level (e.g., near-surface air
temperature), the data is a 3-dimensional array, with dimensions
representing time and two coordinates of location (e.g., latitude and
longitude). Figure \ref{fig:netcdfexample} provides a sketch of the
structure of netCDF files for single-level climate model output. Global
climate models generate output at regularly-spaced time steps, typically
at regularly-spaced grid points around the world. The latitude and
longitude spacing of grid points vary by climate model, but are
typically 1--2 degrees for atmospheric variables in CMIP5 models
\citep{IPCCch9}. Each data point in the netCDF array gives the modeled
value of the variable (e.g., surface temperature) for a single time
point and location. For CMIP5 climate model output, the location units
are in degrees east and degrees north for longitude and latitude,
respectively. For daily output files, the time unit is in days since a
specified origin date-time (e.g., days since 1850-01-01 00:00:00)
\citep{taylor2010cmip5}. All CMIP5 output files are required to include
certain metadata \citep{taylor2010cmip5}, including the experiment,
forcing agents input to the model to create the simulation, time step,
institution and institutional contact information, climate model, and
modeling realm \citep{taylor2010cmip5}. The metadata also must include
units for all of the coordinate variables (e.g., longitude, latitude,
time). The netCDF format allows you to access metadata and variables
describing the dimensions of the data without reading the full file into
memory.

\begin{figure}
\begin{center}
\includegraphics[width = \textwidth]{netcdf_figure}
\end{center}
\caption{Example of structure of a NetCDF climate model output file for a variable reported at a single level, like near-surface air temperature. Data are stored in a three-dimensional array, with measurements at each time step and climate grid location. These data are typically indexed in the netCDF file by longitude, latitude, and time, in that order. For example, if the near-surface air temperature is read into an R object called \code{tas}, you can access the value for the first day at the fourth longitude and third latitude with \code{tas[4, 3, 1]}. In addition to the output variable (temperature in this example), vectors with the ordered values of each dimension (longitude, latitude, and time) can also be read in from the netCDF file, as well as attribute data (e.g., units for variables, the calendar used for time).}
\label{fig:netcdfexample}
\end{figure}

To find out more about the CMIP climate model output data,
\citet{taylor2012overview} and \citet{meehl2007wcrp} are excellent
resources.

\subsection{Working with climate model output in
R}\label{working-with-climate-model-output-in-r}

When working with daily climate model output data, challenges to R users
include: (1) the file format; (2) use of non-Gregorian calendars; and
(3) large file sizes. This section explains these challenges and offers
some strategies for dealing with them.

CMIP5 data are available as netCDF files, for which base R import
functions do not exist. There are, however, a few R packages that can be
used to work with the netCDF file format used for CMIP5 files. Older
packages include \pkg{ncdf} and \pkg{ncvar}, but these do not work with
the newer netCDF version 4 released in 2008 and are no longer available
through CRAN. More recent packages, including \pkg{ncdf4} \citep{ncdf4}
and \pkg{RNetCDF} \citep{michna2013rnetcdf, RNetCDF}, work with both
version 4 and netCDF's older version 3. While climate model output data
for CMIP5 are required to conform with the earlier version (version 3)
\citep{taylor2010cmip5}, it is safer to write code using functions that
can be used with either version, in case future phases of CMIP do not
require files to conform with netCDF version 3.

You can do a number of things with netCDF files in R using these
functions. For example, \pkg{ncdf4}'s \code{nc\_open} function can be
used to open a connection to a netCDF file, and the object returned by
the function includes the file's attribute data. Once a file connection
is open, variables can be read in using the \code{ncvar\_get} function.
For example, the variables defining the dimensions of the sketched
netCDF file in Figure \ref{fig:netcdfexample} could be read into R with
\code{ncvar\_get} with the \code{varid} parameter set to ``lat'',
``lon'', or ``time''. The climate output variable (e.g., near-surface
air temperature) can similarly be read in using \code{ncvar\_get}. In
this case, the \code{varid} parameter should be set using the
appropriate CMIP5 variable name (e.g., ``tas'' for near-surface air
temperature); these variable names can be found in the CMIP Requested
Output tables \citep{taylor2010cmip5}. If you only need data from a
subset of the full file, you can use the dimensional time and location
data to identify the location of the variable data you need in the
netCDF array and use indexing to read only that data into memory,
without needing to read in the full file (for example, with the
\code{nc.get.var.subset.by.axes} function in \pkg{ncdf4.helpers}). Once
the user is done reading in data from the file, the connection can be
closed (e.g., with the \code{nc\_close} function from \pkg{ncdf4}).

\begin{figure}
\begin{center}
\includegraphics[width = \textwidth]{worldmapexample}
\end{center}
\caption{Example of mapping near-surface air temperature data worldwide for a single day of climate model output data. This map uses data from the Geophysical Fluid Dynamics Laboratory's Earth System Model 2G, r1i1p1 ensemble member, on a single day in the summer of 2075. Full code for recreating the map is available in the "starting\_from\_netcdf" vignette of the \pkg{futureheatwaves} package.}
\label{fig:worldmap}
\end{figure}

\begin{figure}
\begin{center}
\includegraphics[width = 0.8\textwidth]{timeseriesexample}
\end{center}
\caption{Example of plotting a time series of temperature simulations between 2071 and 2075 from CMIP5 daily climate model output data for the model grid cell point closest to Beijing, China. This plot uses data from the Geophysical Fluid Dynamics Laboratory's Earth System Model 2G, r1i1p1 ensemble member. Full code for recreating the map is available in the "starting\_from\_netcdf" vignette of the \pkg{futureheatwaves} package.}
\label{fig:timeseries}
\end{figure}

A second challenge when working with climate model output data in R is
that some climate models output to non-Gregorian calendars. Since the
late 1500s, Western dates have been set using the Gregorian calendar,
which has 365.2425-day years. Some climate models, however, are run
using different calendars, including the Julian calendar (365.25-day
years), a calendar where there are no leap years (365-day years), a
calendar where every year is a leap year (366-day years), and a calendar
of twelve 30-day months (360-day years) \citep{cfconventions}. With
these non-Gregorian calendars, R's base functions for converting a
vector to a Date class based on the number of days since an origin date
(\code{as.Date}, \code{as.POSIXct}) do not return the desired values.

Two R packages provide help with non-Gregorian calendars: \pkg{PCICt}
\citep{PCICt} and \pkg{ncdf4.helpers} \citep{ncdf4.helpers}. The
\code{nc.get.time.series} function in \pkg{ncdf4.helpers} pulls and uses
metadata on the calendar stored in the CMIP5 netCDF file's attributes to
convert the ``time'' variable in the file to an object of the
\code{PCICt} class. This class is defined in the \pkg{PCICt} package and
provides Date-like functionality for 360- and 365-day calendars
\citep{PCICt}. However, while these functions will help with handling
most CMIP5 files, the CMIP5 standards allows use of other calendars
which may not be succesfully handled by these functions, so it is
important to assess whether the time variable range in the \code{PCICt}
object correctly matches the expected date ranges for a file when
processing CMIP5 data in R.

Finally, the size of CMIP5 files can make them difficult to work with in
R. CMIP5 climate model output files can be as large as several
gigabytes. The size of the files can therefore be large enough that it
may make more sense to work with smaller chunks of the data in R, rather
than reading all data into memory and working with the data all at once
\citep{RCMIP5}. This problem aggregates when working with multiple
climate models and more than one ensemble member for each of those
climate models.

In addition to these general packages for working with netCDF files,
there are several R packages specifically for working with climate model
output data, including \pkg{RCMIP5} \citep{RCMIP5} and \pkg{wux}
\citep{wux}. However, these packages are more useful for working with
data output at time steps of a month or higher and have limited utility
with the daily climate model output data required for studies of
multi-day extreme events.

The \pkg{RCMIP5} package includes functions to read in CMIP5 data from
netCDF files, scan a directory of CMIP5 files and determine models with
continuous available data, create objects of a special \code{cmip5data}
class to work with CMIP5 data within R, and parse the filenames for all
files in a directory to extract information within the filename. For
this package, most functions only work with monthly or less frequent
data \citep{RCMIP5}. While the \code{loadCMIP5} function does
successfully load daily data as a \code{cmip5data} object, most of the
methods for this object type do not do anything meaningful for daily
data. The package's \code{getFileInfo} function, however, will work with
CMIP5 files of any time step; this function identifies all CMIP5 files
in a directory and creates a data frame with information parsed from the
file name. As a note, the \code{get.split.filename.cmip5} function in
the \pkg{ncdf4.helper} package similarly can be used to parse
information contained in CMIP5 file names \citep{ncdf4.helpers}.

The \pkg{wux} package \citep{wux} includes functions that allow the user
to download CMIP5 monthly-aggregated output directly from within R with
the \code{CMIP5fromESGF} function. The package then uses the
\code{models2wux} function to read in climate model output netCDF files
and convert them to ``WUX'' data frames, which can be used by other
functions in the package. While this function can input climate model
output with daily time steps (the ``what.timesteps'' element of the
\code{modelinput} list input must be set to ``daily''), the function
aggregates this data to a monthly or less frequent (e.g., seasonal)
aggregation when creating the WUX data frame. Therefore, while this
package provides very useful functionality for working with averaged
output of daily climate model output data, it cannot easily be used to
identify and characterize multi-day extreme events like heat waves.

The functions and packages described in this section can be used with
CMIP5 netCDF files to do things in R like map near-surface air
temperatures from a single climate model on a specific day (Figure
\ref{fig:worldmap}) or pull a time series of daily near-surface air
temperature simulations at a specific climate model grid point (Figure
\ref{fig:timeseries}). The ``starting\_from\_netcdf'' vignette that
comes with the ``futureheatwaves'' package provides all code required to
create these figures, as well as more details and code examples on
working with CMIP5 netCDF files in R.

\section{\texorpdfstring{The \pkg{futureheatwaves}
package}{The  package}}\label{the-package}

\subsection{How the package works}\label{how-the-package-works}

\begin{widefigure}
\includegraphics[width = 0.9\textwidth]{OverviewFigure}
\caption{Overview of the functionality of the \pkg{futureheatwaves} package. The package takes a directory with climate projection files (left), for one or more climate models, with one or more ensemble members for each climate model (this example figure shows four climate models with one or two ensemble members each). The \code{gen\_hw\_set} function processes these files to create a data frame for each ensemble member, identifying and characterizing all multi-day extreme events (e.g., heat waves) in the time series projection for that ensemble member. The \code{apply\_all\_models} function allows users to explore these extreme events by applying user-created functions across all the extreme event data frames, creating a summary data frame with results.}
\label{fig:overview}
\end{widefigure}

We created the \pkg{futureheatwaves} package to aid in identifying,
characterizing, and exploring multi-day extreme events in daily climate
model output data. Figure \ref{fig:overview} gives an overview of the
two primary functions of the \pkg{futureheatwaves} package. First, the
\code{gen\_hw\_set} function processes a directory of climate projection
files that are stored locally on the user's computer (Figure
\ref{fig:overview}, ``Climate projections''), to generate a list of all
extreme events in each projection, as well as over a dozen
characteristics of each identified extreme event (Table
\ref{tab:hwcharacteristics}). This package start from files rather than
R objects to avoid loading data from all climate model ensembles at
once; instead, the function loads, processes, and saves output for a
single climate model ensemble member at a time. The extreme events are
identified and characterized at one or more study locations (e.g.,
cities), which the user specifies in an input file. The extreme events
identified for each ensemble member are output as separate files in a
directory specified by the user (Figure \ref{fig:overview}, ``Extreme
events datasets``).

Once the user creates these data frames of location-specific heat waves,
the \code{apply\_all\_models} function can be used to apply custom
functions across all the extreme event data frames generated by
\code{gen\_hw\_set}. This functionality allows users to create summaries
of extreme events across all climate models and ensemble members (Figure
\ref{fig:overview}, right). The function can be used to generate summary
statistics (e.g., determine average heat wave length or total frost
days) or to apply more complex functions (e.g., apply epidemiologic
effect estimates across the heat waves to generate health impact
estimates).

When using this package, CMIP5 climate model output data require some
pre-processing. Data will need to be saved in a specific format, with
files stored in a specific directory structure. Full details of the
required file and directory structure are provided in the package's
``futureheatwaves'' vignette, while tips and an R script for conducting
this processing starting from CMIP5 netCDF files are given in the
``starting\_from\_netcdf'' vignette.

This package can be used for study locations worldwide. The
``starting\_from\_netcdf''" package vignette provides an example of
using this package to identify and explore future heat waves in several
Chinese cities.

\begin{table}
\newcolumntype{L}[1]{>{\raggedright\arraybackslash}p{#1}}
\begin{tabular}{lL{\dimexpr0.8\textwidth-2\tabcolsep\relax}}
\toprule
Column name & Description of characteristic \\
\midrule
\code{mean.var} & Average daily value of the variable across all days in the extreme event, in the units in which the variable is expressed in input files (e.g., average daily mean temperature during the heat wave in degrees Kelvin) \\
\code{max.var} & Highest daily value of the variable across all days in the extreme event, in the units in which the variable is expressed in input files \\
\code{min.var} & Lowest daily value of the variable across all days in the extreme event, in the units in which the variable is expressed in input files \\
\code{length} & Number of days in the event \\
\code{start.date} & Date of the first day of the event \\
\code{end.date} & Date of the last day of the event \\
\code{start.doy} & Day of the year of the first day of the event (1 = Jan. 1, etc.)\\
\code{start.month} & Month in which the event started (1 = January) \\
\code{days.above.abs.thresh.1} & Number of days in the event above a specified absolute threshold (default is the number of days in the event above 80\textsuperscript{o}F / 26.7\textsuperscript{o}C, but this and the following three absolute thresholds can be changed with the \code{absolute\_thresholds} argument in \code{gen\_hw\_set}) \\
\code{days.above.abs.thresh.2} & Number of days in the event above a specified absolute threshold (default is the number of days in the event above 85\textsuperscript{o}F / 29.4\textsuperscript{o}C) \\ 
\code{days.above.abs.thresh.3} & Number of days in the event above a specified absolute threshold (default is the number of days in the event above 90\textsuperscript{o}F / 32.3\textsuperscript{o}C) \\
\code{days.above.abs.thresh.4} & Number of days in the event above a specified absolute threshold (default is the number of days in the event above 95\textsuperscript{o}F / 35.0\textsuperscript{o}C) \\
\code{days.above.99th} & Number of days in the event above the 99\textsuperscript{th} percentile of the variable for the location, using the period specified with the \code{referenceBoundaries} argument in \code{gen\_hw\_set} as a reference for determining these percentiles \\
\code{days.above.99.5th} & Number of days in the event above the 99.5\textsuperscript{th} percentile of the variable for the location, using the period specified with the \code{referenceBoundaries} argument in \code{gen\_hw\_set} as a reference for determining these percentiles \\
\code{first.in.year} & Whether the event was the first to occur in its calendar year in the location \\
\code{mean.var.quantile} & The percentile of the average variable value during the event compared to the location's year-round distribution of the variable, based on the variable distribution for the location during the period specified by the \code{referenceBoundaries} argument in \code{gen\_hw\_set} \\
\code{max.var.quantile} & The percentile of the maximum variable value during the event compared to the location's year-round distribution of the variable, based on the variable distribution for the location during the period specified by the \code{referenceBoundaries} argument in \code{gen\_hw\_set} \\
\code{min.var.quantile} & The percentile of the minimum variable value during the event compared to the location's year-round distribution of the variable, based on the variable distribution for the location during the period specified by the \code{referenceBoundaries} argument in \code{gen\_hw\_set} \\
\code{mean.seasonal.var} & The location's average seasonal value of the variable (by default, season is set to May--September, but this can be changed with the \code{seasonal\_months} argument in \code{gen\_hw\_set}), based on the variable values for the location during the years specified by the \code{referenceBoundaries} argument in \code{gen\_hw\_set} \\
\code{mean.yearround.var} & The location's average year-round value of the variable, based on the variable values for the location during the years specified by the \code{referenceBoundaries} argument in \code{gen\_hw\_set} \\
\bottomrule
\end{tabular}
\caption{Extreme event characteristics measured by the \code{gen\_hw\_set} function in the \pkg{futureheatwaves} package. The left column gives the name of each variable's column in the extreme event datasets created by the \code{gen\_hw\_set} function. When characterizing extreme events below a threshold, like cold spells, appropriate alternatives are given for some columns (e.g., \code{days.below.abs.thresh.1}, \code{days.below.1st}).}
\label{tab:hwcharacteristics}
\end{table}

\subsection{Example data}\label{example-data}

We have included data files in the package to serve as example files so
that users can try this package before applying it to their own
directory of climate projection files. These example data come from two
climate models that are a part of CMIP5: (1) the model of the Beijing
Climate Center, China Meteorological Administration (BCC)
\citep{xin2013introduction} and (2) the National Center for Atmospheric
Research's (NCAR's) Community Climate System Model, version 4 (CCSM4)
\citep{gent2011community}. We include one ensemble member from BCC
(r1i1p1) and two from CCSM (r1i1p1 and r2i1p1). Once the
\pkg{futureheatwaves} package is installed and loaded, the user can find
the local location of these files using R's \code{system.file} function.

To ensure that the size of this example data is reasonably small, we
have only included projection data for grid points from these climate
models that are near five U.S. east coast cities: New York, NY;
Philadelphia, PA; Newark, NJ; Baltimore, MD, and Providence, RI.
Further, to keep the file sizes reasonably small, the historical
projections range over the years 1990 to 1999, while the future
projections are limited to 2060 to 2079. Users' applications of this
package will likely use directories with many more climate model
ensemble members and more locations; however, the operation of the
package is the same for this smaller example application as it would be
for a much larger application.

\subsection{\texorpdfstring{Basic example of using
\pkg{futureheatwaves}}{Basic example of using }}\label{basic-example-of-using}

Once climate model output files are set up as specified in the
``futureheatwaves'' package vignette, the package can process them to
identify and characterize heat waves in each ensemble member's
projection for each location using the \code{gen\_hw\_set} function. For
example, to process the example climate model output data included with
the package, the user can run:

\begin{Schunk}
\begin{Sinput}
projection_dir_location <- system.file("extdata/cmip5",
                                       package = "futureheatwaves")
city_file_location <- system.file("extdata/cities.csv",
                                  package = "futureheatwaves")

gen_hw_set(out = "example_results",
           dataFolder = projection_dir_location ,
           dataDirectories = list("historical" = c(1990, 1999),
                                        "rcp85" = c(2060, 2079)),
           citycsv = city_file_location,
           coordinateFilenames = "latitude_longitude_NorthAmerica_12mo.csv",
           tasFilenames = "tas_NorthAmerica_12mo.csv",
           timeFilenames = "time_NorthAmerica_12mo.csv")
\end{Sinput}
\end{Schunk}

This code first identifies and saves as objects the path names on the
user's computer of the example climate projections directory
(\code{projection\_dir\_location}) and the file of study locations
(\code{city\_file\_location}). The \code{gen\_hw\_set} function
processes the example input and creates a new directory,
\file{example\_results}, with files of identified and characterized heat
waves, in the user's current working directory. In this example code,
the processing is done using default values for the event definition,
years for which to generate the heat wave data sets, etc. How and why to
customize these choices are explained later in the text. Function
arguments (e.g., \code{dataDirectories}, \code{tasFilenames}) are used
to specify the format of the data and the directory structure.

Once the function has completed running, results will be written locally
to the directory specified by the \code{out} argument of
\code{gen\_hw\_set}. This directory will include files with some basic
information about the climate models and the closest grid points of each
climate model to each location, as well as a directory with files of
identified and classified extreme events for each ensemble member,
including all characteristics in Table \ref{tab:hwcharacteristics}. See
the package's vignettes for more details on the content and structure of
this output.

Once the user has created a directory of characterized event files for
each ensemble member (``Extreme event data sets'', Figure
\ref{fig:overview}), he or she can explore the results using the
\code{apply\_all\_models} function. This function allows the user to
apply custom R functions across all extreme event data frames created by
the \code{gen\_hw\_sets} call. The user can apply any R function that
follows certain standards in accepting input and returning output. Full
details on these standards are given in the \pkg{futureheatwaves}
package vignette.

As an example, if the user wanted to calculate the average temperature
of the heat waves identified for each ensemble member, he or she could
write a simple function:

\begin{verbatim}
average_mean_temp <- function(hw_datafr){
        out <- mean(hw_datafr$mean.var)
        return(out)
        }
\end{verbatim}

\noindent This function could then be applied across all extreme event
data sets output by \code{gen\_hw\_set} using the
\code{apply\_all\_models} function. For example, to apply this function
to all the example output results that come with the package, the user
could run:

\begin{Schunk}
\begin{Sinput}
out <- system.file("extdata/example_results", package = "futureheatwaves")
apply_all_models(out = out, FUN = average_mean_temp)
\end{Sinput}
\begin{Soutput}
#>   model ensemble    value
#> 1  bcc1        1 302.3745
#> 2  ccsm        1 302.4458
#> 3  ccsm        2 302.3428
\end{Soutput}
\end{Schunk}

This output gives the results (\code{value} column) of running the
custom function for each ensemble member of each climate model. Note
that the location of the directory with the heat wave data frames must
be specified using the \code{out} argument when calling
\code{apply\_all\_models}. Typically, this will be the directory path
for the directory specified with the \code{out} argument in
\code{gen\_hw\_set}.

Location-specific results can be generated using the
\code{city\_specific} argument in \code{apply\_all\_models}:

\begin{Schunk}
\begin{Sinput}
apply_all_models(out = out, FUN = average_mean_temp, city_specific = TRUE)
\end{Sinput}
\begin{Soutput}
#>    model ensemble city    value
#> 1   bcc1        1 balt 305.1816
#> 2   bcc1        1  nwk 300.3367
#> 3   bcc1        1   ny 300.3367
#> 4   bcc1        1 phil 305.1816
#> 5   bcc1        1 prov 298.0402
#> 6   ccsm        1 balt 303.1277
#> 7   ccsm        1  nwk 302.4053
#> 8   ccsm        1   ny 302.4053
#> 9   ccsm        1 phil 302.3425
#> 10  ccsm        1 prov 301.8895
#> 11  ccsm        2 balt 302.9373
#> 12  ccsm        2  nwk 302.2748
#> 13  ccsm        2   ny 302.2748
#> 14  ccsm        2 phil 302.2858
#> 15  ccsm        2 prov 301.9520
\end{Soutput}
\end{Schunk}

This output is structured as ``tidy" data \citep{wickham2014tidy},
allowing it to be used easily with the graphing package
\CRANpkg{ggplot2} \citep{ggplot2}. This functionality can be used to
create a number of other summaries of the identified extreme events. For
example, it could be used to determine average heat wave length or
estimate how much earlier in the year events are expected to start
across an ensemble of climate model simulations. The functionality can
also be used for more complex analysis of extreme event files. For
example, it can be used to apply epidemiological models of heat wave to
estimate excess heat-related mortality under different future scenarios;
an example of this application is provided in the ``futureheatwaves''
vignette that comes with the package.

\subsection{Customizing the extreme event
definition}\label{customizing-the-extreme-event-definition}

By default, the package identifies extreme events in climate model
output data using a specific definition for heat waves that has been
used in some epidemiological and climate impact research (e.g.,
\citet{anderson2009weather}):

\begin{quote}
A \dfn{heat wave} is two or more days at or above a city-specific
threshold temperature, with the threshold determined as the
98\textsuperscript{th} percentile of year-round temperature in the city
during some reference period (by default, 1990--1999).
\end{quote}

\noindent However, this is not the only accepted heat wave definition in
the scientific literature. A variety of different heat wave definitions
have been used to identify heat waves in a time series of temperature
data
\citep{smith2013heat, kent2014heat, chen2015influence, anderson2009weather},
and the choice of heat wave definitions can influence both projected
heat wave trends \citep{smith2013heat} and estimates of health risks
during events
\citep{chen2015influence, kent2014heat, anderson2009weather}. Further,
other types of extreme events will be defined differently than heat
waves (for example, frost day spells may be defined as one or more days
with temperature at or below 32\textsuperscript{o}F /
0\textsuperscript{o}C).

Therefore, this package allows the user to extensively customize the
definition used to identify extreme events. Users can write a custom R
function with either a different heat wave definition (see
\citet{smith2013heat} and \citet{kent2014heat} for listings of some of
the definitions used in scientific studies) or with a definition
appropriate for a different type of extreme event (e.g., one or more
days at or below 32\textsuperscript{o}F / 0\textsuperscript{o}C for
frost day spells). For heat wave identification, researchers might want
to use a specific definition, for example, because it matches the
definition used by local health officials to declare heat wave warnings
or, in the case of health impact assessments, to match with a definition
used in an epidemiological study. For studies of other extreme events, a
heat wave definition likely will not be applicable and so a customized
definition is necessary.

Three components of the extreme event definition can be easily
customized in the \code{gen\_hw\_set} function call, without creating a
new R function to use to identify heat waves. First, many extreme event
definitions are based on conditions that are rare in the study location
\citep{IPCCch1}, but may vary in how rare conditions must be for a
period to qualify as an extreme event. For example, some of the
different definitions used to identify heat waves vary only in the
percentile temperature used for a threshold (e.g., one definition is
\(\ge2\) days at or above the 98\textsuperscript{th} percentile
temperature at a location while another is \(\ge2\) days at or above the
99\textsuperscript{th} percentile temperature;
\citet{kent2014heat, smith2013heat}). Therefore, the
\pkg{futureheatwaves} package allows users to change the percentile of
the variable of interest required for an extreme event using the
\code{probThreshold} option in \code{gen\_hw\_set}. Other heat wave
definitions vary only in the number of consecutive days that must be
over the threshold for a period to quality as an extreme event (e.g.,
one definition is \(\ge2\) days at or above the 98\textsuperscript{th}
percentile temperature at a location while another is \(\ge4\) days at
or above the 98\textsuperscript{th} percentile temperature;
\citet{anderson2009weather}). Therefore, the package allows the user to
change the number of days used in the heat wave definition using the
\code{numDays} argument in the \code{gen\_hw\_set} function. Combined,
these two customization choices allow the user to identify heat waves
using many of the heat wave definitions used in previous climate and
health research-- for example, 9 of the 16 heat wave definitions used in
\citet{kent2014heat} could be fit using different combinations of these
two options for specifying threshold percentile and number of days.
Third, some extreme events like cold waves and frost day spells are
defined as a certain number of days below, rather than above, a
threshold. While the default is to identify events by searching for days
above a threshold, this behavior can be changed with the
\code{above\_threshold = FALSE} argument in the
\code{gen\_hw\_set function}.

Beyond these simpler options, the customization of the event definition
is even more extensive as one has the option of writing and using a
custom R function to identify extreme events. This functionality allows
the user to use definitions that either require a number of days above
an absolute threshold (e.g., maximum temperature of \(\ge\)
95\textsuperscript{o}F for \(\ge1\) day
\citet{kent2014heat, tan2007heat}; minimum temperature \(\le\)
0\textsuperscript{o}C for \(\ge1\) day for frost day spells) or that
require a combination of thresholds to be met (e.g., maximum daily
temperature above a lower threshold every day of the heat wave and above
a higher threshold for a certain number of days within the heat wave;
\citet{kent2014heat, peng2011toward}). To use a customized event
definition, the user must write and load an R function that implements
the definition. This custom function is passed to the
\code{gen\_hw\_set} function using the \code{IDheatwavesFunction}
argument. To work correctly, this custom function must allow only
specific inputs and generate only specific outputs; details about the
required structure are provided in the \pkg{futureheatwaves} package
vignette. To increase processing speed when identifying extreme events,
we coded parts of the default event definition function in C++ and
synced it with R using the \CRANpkg{Rcpp} package \citep{Rcpp}. Users
should consider a similar strategy for custom heat wave definitions,
especially if processing a large number of climate projection files.

\subsection{Exploring adaptation
scenarios}\label{exploring-adaptation-scenarios}

Extreme events are often based on conditions that are rare for a
specific location using location-specific relative thresholds. These
thresholds are often defined for climate impact studies based on a
variable's distribution at that location in present-day or historical
data. However, it can be interesting to explore trends in extreme events
under climate change if extreme events are identified based on variable
distributions during the projection period, or another future period.
For some extreme events, impacts are associated with how rare the
conditions during the event are compared to norms in the location
\citep{anderson2009weather}. This relationship raises the question of
whether extreme events should be defined using a percentile threshold
based on present-day variable distributions or based on distributions in
the time period being projected. The \pkg{futureheatwaves} package
allows users to specify the time period to use when determining a
relative threshold for an event definition using the
\code{thresholdBoundaries} argument in the function \code{gen\_hw\_set},
allowing users to explore how sensitive projections of impacts are to
this choice of the time period to use when determining relative variable
measures, including thresholds used for percentile-based event
definitions.

Similarly, some of the event characteristics (e.g.,
\code{mean.temp.quantile}, Table \ref{tab:hwcharacteristics}) are also
calculated by the package based on relative temperature, providing
measures of how the value of the variable of interest during an extreme
event compares to the typical distribution of that variable at that
location (e.g., the ``mean.var.quantile'', ``min.var.quantile'', and
``max.var.quantile'' characteristics, Table \ref{tab:hwcharacteristics})
or how long conditions of a certain rarity persisted during the event
(e.g., the ``days.above.99th'' and ``days.above.99.5th''
characteristics, Table \ref{tab:hwcharacteristics}). These
characteristics are measured for each of the extreme events identified
by the \code{gen\_hw\_set} function by taking the absolute value of the
variable during the event (e.g., average temperature during the heat
wave is 90\textsuperscript{o}F ~32.2\textsuperscript{o}C) and comparing
it to the location's typical variable distribution. This process
generates relative measures of how intense the event is compared to what
is normal in that location (e.g., 90\textsuperscript{o}F
~32.2\textsuperscript{o}C is in the 99\textsuperscript{th} percentile of
year-round temperatures in the location).

These relative event characteristics will vary depending on whether you
calculate them based on a location's present-day variable distribution
or on the location's variable distribution in the future, since the
distributions of many relevant variables (e.g., temperature,
precipitation) are expected to change in many locations with climate
change. The package therefore allows the user to specify date ranges of
the temperature distributions to be used in calculating these relative
temperature metrics in each location, which can be done using the
\code{referenceBoundaries} option of \code{gen\_hw\_set}.

\subsection{Mapping grid points}\label{mapping-grid-points}

\begin{figure}
\begin{center}
\includegraphics[width = 0.9\textwidth]{ExampleLeaflet}
\end{center}
\caption{Snapshot of an interactive map created using the \code{map\_grid\_leaflet} showing the locations of study cities and their matching climate model grid points for the BCC climate model example data included with \pkg{futureheatwaves}. The lines on the map connect each climate model grid point to the study location(s) for which that grid point was used. The interactive maps include pop-ups with city identiers; one is shown open in this snapshot as an example. From this map, you can see that the climate model grid point closest to New York City for this climate model is over the Atlantic Ocean.}
\label{fig:gridmap}
\end{figure}

It can be useful to explore the location of the climate model grid point
used to pull climate model output for each study location with a given
climate model. Therefore, the package has a function called
\texttt{map\_grid\_leaflet} that plots the locations of grid points used
for each location from each climate model. This function is built using
the htmlWidget \pkg{leaflet} package \citep{leaflet}. The following code
illustrates the use of this function with the example data to create
Figure \ref{fig:gridmap}, which plots the grid points used in the
example data from the BCC climate model in the example data:

\begin{Schunk}
\begin{Sinput}
out <- system.file("extdata/example_results", package = "futureheatwaves")
map_grid_leaflet(plot_model = "bcc1", out = out)
\end{Sinput}
\end{Schunk}

\noindent This interactive map can be panned and zoomed to explore the
locations of climate model grid points used to represent each study
location. This mapping function works for study locations worldwide.

\subsection{Extensions}\label{extensions}

While this package was created to be used for research on extreme events
in climate change projections, it can be used more broadly. For example,
there are other episodes like wildfires and air pollution where it may
be interesting to identify extended periods of high exposures in
projection time series, and this package could be applied to gridded air
pollution model output to explore these exposures.

\section{Acknowledgements}\label{acknowledgements}

This work was supported by grants from the National Institute of
Environmental Health Sciences (R00ES022631), the National Science
Foundation (1331399), and the Colorado State University Vice President
for Research. Thank you to early testers of the package: Julia
Bromberek, Wande Benka-Coker, Josh Ferreri, Ryan Gan, Molly Gutilla,
Mike Lyons, Casey Quinn, Rachel Severson, and Meilin Yan.

\bibliography{Anderson}

\address{%
G. Brooke Anderson\\
Colorado State University\\
Department of Environmental \& Radiological Health Sciences\\ 1681 Campus Delivery\\ Fort Collins, Colorado 80523\\
}
\href{mailto:brooke.anderson@colostate.edu}{\nolinkurl{brooke.anderson@colostate.edu}}

\address{%
Colin Eason\\
Colorado State University\\
Department of Computer Science\\ 1873 Campus Delivery\\ Fort Collins, Colorado 80523\\
}
\href{mailto:aimesce@gmail.com}{\nolinkurl{aimesce@gmail.com}}

\address{%
Elizabeth A. Barnes\\
Colorado State University\\
Department of Atmospheric Science\\ 1371 Campus Delivery\\ Fort Collins, CO 80523\\
}
\href{mailto:eabarnes@atmos.colostate.edu}{\nolinkurl{eabarnes@atmos.colostate.edu}}

